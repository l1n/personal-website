\documentclass{article}
\title{Cereal, The Farmer's Friend}
\author{KR44057}
\usepackage{graphicx}
\usepackage{hyperref}

\begin{document}
\maketitle
\begin{abstract}
This week's journal is on the \textit{Homeric Hymn to Demeter}. The Hymn is one of the more powerful pieces of ancient literature. It is directly tied to actual Greek practices of worship, and it is revelatory about several different cultural expectations. For this fifth reading journal, consider the goddess Demeter - how do her actions in the hymn reflect her position as a goddess of the earth, fertility, and the harvest. Choose four specific instances and explain your choices.
\end{abstract}
Demeter (or Ceres) is one of the oldest of the Greek gods, and one of the most revered - in an agrarian society it is natural that the most important deities would be those whose whims determine literal life or death. In the Homeric Hymnn to Demeter, she is called the "queen among goddesses" (118), and her Elusian Mysteries were those sought by many major heroes before entering the underworld, including Herakles (see \href{https://lin.noblejury.com/umbc/myth/herakles-12.txt}{Art Blog \#2}).

\paragraph{Her mood is directly connected to the growth of crops.} Rhea bids Demeter after the return of Persephone to "increase ... the fruit that gives [men] life" (469), and Demeter (though she had hidden crops before) "straightway made fruit to spring up from the rich lands, so that the whole wide earth was laden with leaves and flowers" (470).
\paragraph{She is a protector of children.} The boon sought by Metaneira from the goddess was that of the health of her son, and to that Demeter responds favorably: "Gladly will I take the boy to my breast, as you bid me, and will nurse him" (226).
\paragraph{She walks among us.} When Demeter is sadenned by the loss of her daughter, she wanders the earth (94). While Metaneira is in awe of her (190) at their meeting, there is little surprise that the woman that her daughters bring to her is in fact the immortal goddess, and she is quick to ask for a boon (220).
\paragraph{She is a secret goddess.} It is clear that this hymn is important to the origin of the Elusian Mysteries, and they are mysteries for a reason: "Happy is he among men upon earth who has seen these mysteries; but he who is uninitiate and who has no part in them, never has lot of like good things once he is dead, down in the darkness and gloom" (480).

\begin{figure}
	\centering
	\includegraphics[width=4in]{C:/Users/sda1/Pictures/cyber-corn.jpg}
	\caption{Demeter, as imagined. Shutterstock Stock photo ID: 42829648}
\end{figure}

\href{http://data.perseus.org/citations/urn:cts:greekLit:tlg0013.tlg002.perseus-eng1:2}{Translation by Thomas W. Allen, E. E. Sikes, 1904}
\end{document}
